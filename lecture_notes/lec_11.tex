\section{Subsequences}

\lecture{11}{2021-09-24}{}

\begin{defn}{Subsequence}{subsequence}
	Given a sequence \((a_n)\) and a strictly monotone increasing sequence of natural numbers \((n_i)\), the sequence \((a_{n_i})\) is called a \emph{subsequence} of  \((a_n)\).

In other words, we can say that \((b_k)\) is a subsequence of \((a_n)\) if there exists a strictly monotone increasing \(f: \mathbb{N} \to \mathbb{N}\) so that \(b_k = a_{f(k)}\) for all \(k\).
\end{defn}

\begin{thm}{}{seqconvsubseqconv}
	Let \(X\) be a metric space.
	A sequence of elements in \(X\) converges to \(L \in X\) if, and only if, every of its subsequences converges to \(L \in X\).
\end{thm}

\begin{dem}{}{}
	The inverse implication is straightforward, since the sequence is a subsequence of itself. Let's prove the direct implication.
	Let \((a_n)\) be a sequence so that \(a_n \to L\). Let \((a_{n_i})\) be a subsequence of \((a_n)\).
	Let \(\epsilon > 0\). Since \(a_n \to L\), there exists \(N\) so that \[
		d(L, a_n) < \epsilon,
	\] for all \(n \geq N\). Note that \(n_i \geq i\). Therefore, for the same choice of \(N\), it holds that \[
		d(L, a_{n_i}) < \epsilon
	\] for all \(i \geq N\).
	Therefore,  \(a_{n_i} \to L\).
\end{dem}
