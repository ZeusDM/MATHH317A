\lecture{23}{2021-10-29}{}

\subsection{The Cantor set}

\begin{defn}{Cantor set}{cantorset}
	Let \(E_0\) be the interval \([0, 1]\). Remove the middle third open interval to obtain \(E_1\), which therefore can be written as \[
		[0, \tfrac{1}{3}] \cup [\tfrac{2}{3}, 1].
	\]
	Remove the middle third open intervals to obtain \(E_2\), which therefore can be written as \[
		[0, \tfrac{1}{9}] \cup
		[\tfrac{2}{9}, \tfrac{1}{3}] \cup
		[\tfrac{2}{3}, \tfrac{7}{9}] \cup
		[\tfrac{8}{9}, 1].
	\]
	Define \(E_3, E_4, \dots\) analogously.
	The cantor set \(\mathcal C\) is defined as the intersection of all sets \(E_n\), i.e., \[
		\mathcal C = \bigcap_{n=0}^\infty E_n.
	\]
\end{defn}

\begin{prop}{}{}
	The Cantor set \(\mathcal C\) is nowhere dense, i.e., \(\mathring{\overline{\mathcal C}} = \varnothing\).
\end{prop}

\begin{thm}{}{}
	The Cantor set is uncountable.
\end{thm}

\begin{prop}{}{}
	The Cantor set is the set of all numbers \(x\) that can be written as \[
		x = (0.a_1a_2a_3\dots)_3 = \sum_{n = 1}^\infty \frac{a_n}{3^n}
	\]
	with \(a_i \in \{0, 2\}\).
\end{prop}
