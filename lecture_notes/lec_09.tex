\lecture{9}{2021-09-20}{}

\begin{defn}{Boundness}{seqbounded}
	A sequence \((a_n)_{n\in\mathbb{N}}\) is bounded if there exists \(M \in \mathbb{R}\) so that \(|a_n| \leq M\) for all \(n \in \mathbb{N}\).
\end{defn}

\begin{thm}{A convergent sequence is bounded}{}
	If \((a_n)_{n\in\mathbb{N}}\) is a convergent sequence, then \((a_n)\) is bounded.
\end{thm}

\begin{dem}{}{}
	Let \(L\) be the limit of such sequence.
	Let \(\epsilon = 1\).
	Then, there exists \(N \in \mathbb{N}\) so that \(|a_n - L| < 1\) for all \(n \geq N\).
	Triangle inequality implies that \(|a_n| < |L| + 1\) for all \(n \geq N\).
	Define  \[
		M = \max\{|a_1| + 1, |a_2| + 1, \dots, |a_{N-1}| + 1, |L| + 1\}.
	\]
	Then, for this choice of \(M\), it holds that \(|a_n| < M\) for all \(n \in \mathbb{N}\).
	Therefore, \((a_n)\) is bounded.
\end{dem}

