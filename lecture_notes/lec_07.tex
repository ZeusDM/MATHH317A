\lecture{7}{2021-09-15}{Limits}

\begin{dem}{using Cantor's diagonalization}{}
	We'll prove \((0, 1)\) is uncountable, which implies \(\mathbb{R}\) is uncountable.

	Assume \((0, 1)\) is countable, therefore, there exists a bijective function \(f: \mathbb{N} \to (0, 1)\).

	Let's write out decimal expansions\footnote{What are decimal expansions? We only need to know that decimal expansions are unique except for some duplication, like \(0.09999 = 0.1\).} of \(f(1), f(2), \dots\). If there's doubt between a recurrent \(9\) or a recurrent \(0\) in the end, we choose the latter form. We write
	\begin{align*}
		f(i) &= 0.a_{i1}a_{i2}a_{i3}\dots,\\
	\end{align*}
	with \(a_{ij} \in \{0, 1, 2, 3, 4, 5, 6, 7, 8, 9\}\).
	Let \(b_k = 1\), if \(a_{kk}\) is odd, and \(b_k = 2\), if \(a_{kk}\) is even. Note that \(c_{k} \neq b_{kk}\) and \(c_{k} \notin \{0, 9\}\) for all \(k\).
	Therefore, \(x = 0.b_1b_2b_3\dots\) cannot be on the image of \(f\); a contradiction.
\end{dem}

Another perspective on the Cantor's proof arises by using the binary base, instead of the decimal base. For each real number \(x = 0.x_1x_2x_3\dots\), we can define a \(f(x) = \{n \in \mathbb{N} : a_n = 1\}\). This is almost\footnote{The same number with two expansions yields a problem.} a bijection because, but nevertheless, we can conclude that, in some sense, \[
		|\mathbb{R}| = 2^{|\mathbb{N}|}.
\]

\chapter{Normed Vector Spaces and Metric Spaces}

\section{Complex Numbers}

\begin{defn}{Complex number}{complexnumber}
	The set of complex numbers, denoted by \(\mathbb{C}\), is the set of pairs \((a, b)\) of real numbers. On top of that, we define addition and multiplication of complex numbers by
	\begin{enumerate}[label = \textbullet]
		\item \((a, b) + (c, d) = (a + c, b + d)\).
		\item \((a, b) \cdot (c, d) = (ac - bd, ad + bc)\).
	\end{enumerate}
\end{defn}

\begin{prop}{\(\mathbb{C}\) is a field}{cisafield}
	\((\mathbb{C}, +, \cdot)\) is a field.
\end{prop}

Consider the complex numbers of the form \((a, 0)\).
Note that \((a, 0) + (a', 0) = (a + a', 0)\) and  \((a, 0) \cdot (a', 0) = (aa', 0)\).
Therefore, this subset of the complex numbers is isomorphic to \(\mathbb{R}\).
Similarly as we've seen in previous chapters, we can say that \(\mathbb{R} \subset \mathbb{C}\), referring to this natural homomorphism; i.e., if \(a\) is a real number, then we'll also use \(a\) to talk about the complex number \((a, 0)\).

\begin{defn}{Imaginary unit}{imaginaryunit}
	Let \(i = (0, 1) \in \mathbb{C}\).
\end{defn}

\begin{prop}{}{}
	Let \((a, b) \in \mathbb{C}\). Then, \[
		(a, b) = a + bi.
	\]
\end{prop}

\begin{dem}{}{}
	\begin{align*}
		a + bi &= (a, 0) + (b, 0)(0, 1) \\
			   &= (a, 0) + (0, b) \\
			   &= (a, b).
	\end{align*}
\end{dem}

\begin{defn}{Conjugate}{conjugate}
	Given real numbers \(a, b\) and a complex number \(z = a + bi\), we define \emph{the conjugate of \(z\)} as \(a - bi\), denoted by \(\overline z\).
\end{defn}

\begin{defn}{Real and imaginary part}{realimaginarypart}
	Given real numbers \(a, b\) and a complex number \(z = a + bi\), the numbers \(a\) and \(b\) are called \emph{the real part} and \emph{the imaginary part of \(z\)}, respectively. Symbolically, we can write \[
		\mathrm{Re}(z) = a, \qquad \mathrm{Im}(z) = b.
	\]
\end{defn}

\begin{prop}{}{}
	Given \(z, w \in \mathbb{C}\), we have
	\begin{enumerate}[label = \textbullet]
		\item \(\overline{z + w} = \overline z + \overline w\).
		\item \(\overline{zw} = \overline z \cdot \overline w\).
		\item \(z + \overline z = 2 \mathrm{Re}(z)\) and  \(z - \overline{z} = 2i \mathrm{Im}(z)\).
		\item \(z \overline z\) is a non-negative real number.
	\end{enumerate}
\end{prop}

\begin{defn}{Absolute value of a complex number}{absolutevaluecomplex}
	If \(z = a + bi\) is a complex number, then we define \(|z| = (z\overline z)^{1/2} = (a^2 + b^2)^{1/2}\).
\end{defn}

\section{Euclidean Spaces}

\begin{defn}{Euclidean Space}{euclideanspace}
	Given an positive integer \(n\), let \(\mathbb{R}^n\) be the set of all \(n\)-uples \[
		\mathbf x = (x_1, x_2, \dots, x_n)
	\] of real numbers. If \(\mathbf x = (x_1, x_2, \dots, x_n)\) and \(\mathbf y = (y_1, y_2, \dots, y_n)\) are in \(\mathbb{R}^n\), and \(\alpha \in \mathbb{R}\), we define
	\begin{align*}
		\mathbf x + \mathbf y &= (x_1 + y_1, x_2 + y_2, \dots, x_n + y_n) \\
		\alpha \mathbf x &= (\alpha x_1, \alpha x_2, \dots, \alpha x_n).
	\end{align*}
	These definitions make the set \(\mathbb{R}^n\) into a vector space over the real field.

	We further define the \emph{inner product} by \[
		\mathbf x \cdot \mathbf y = \sum_{i=1}^n x_iy_i
	\]
	and the \emph{norm} of \(\mathbf x\) by \[
		|\mathbf x| = (\mathbf x \cdot \mathbf x)^{1/2}.
	\]

	This structure (\(\mathbb{R}^n\), equipped with the inner product and norm) is called the Euclidean \(n\)-dimensional space.
\end{defn}

\begin{prop}{\(\mathbb{R}^n\) is a metric space}{}
	Define \(d \colon \mathbb{R}^n \times \mathbb{R}^n \to \mathbb{R}\) by \(d(\mathbf x, \mathbf y) = |\mathbf x - \mathbf y|\). \((\mathbb{R}^n, d)\) is a metric space.
\end{prop}

\section{Normed Vector Space}

\begin{defn}{Normed Vector Space}{normedvectorspace}
	Let \(F\) be either \(\mathbb{R}\) or \(\mathbb{C}\).

	A \emph{normed vector space} is a \emph{vector space} \(W\) over \(F\), equipped with a norm \(||\textbullet||\colon W \to \mathbb{R}\), satistying the following conditions:
	\begin{enumerate}
		\item \(||v|| \geq 0\) for all \(v \in W\).
		\item \(||v|| = 0\) if, and only if, \(v = 0\).
		\item \(||cv|| = |c| \cdot ||v||\) for all \(v \in W\) and for all \(c \in F\).
		\item \(||x + y|| \leq ||x|| + ||y||\) for all \(x, y \in W\).
	\end{enumerate}
\end{defn}

\begin{exmp}{}{}
	\(\mathbb{R}\) is a normed vector space, considering the norm \(|x|\).
	\(\mathbb{C}\) is a normed vector space, considering the norm \(|z|\).
\end{exmp}

\subsection{Euclidean Spaces}

\begin{defn}{Euclidean Space}{euclideanspace}
	Given an positive integer \(n\), consider the vector space \(\mathbb{R}^n\).
	If \(\mathbf x = (x_1, x_2, \dots, x_n)\) and \(\mathbf y = (y_1, y_2, \dots, y_n)\), we define the \emph{inner product} by \[
		\mathbf x \cdot \mathbf y = \sum_{i=1}^n x_iy_i
	\]
	and the \emph{norm} of \(\mathbf x\) by \[
		||\mathbf x|| = (\mathbf x \cdot \mathbf x)^{1/2}.
	\]

	This structure is called the Euclidean \(n\)-dimensional space.
\end{defn}

\begin{prop}{}{}
	The Euclidean \(n\)-dimensional space is a normed vector space.
\end{prop}

\section{Metric Space}

\begin{defn}{Metric Space}{metricspace}
	A \emph{metric space} is a pair \((X, d)\), where \(X\) is a set and \(d\colon X \times X \to \mathbb{R}\) is a function, called \emph{metric}, that satisfies:
	\begin{enumerate}[label = \textbullet]
		\item \(d(x, y) \geq 0\) for all \(x, y \in X\); with equality if, and only if, \(x = y\).
		\item \(d(x, y) = d(y, x)\) for all \(x, y \in X\). 
		\item \(d(x, y) \leq d(x, z) + d(z, y)\) for all \(x, y, z \in X\).
	\end{enumerate}
\end{defn}

\begin{exmp}{}{}
	Consider \(d(x, y) = |x - y|\), in all following examples (with the appropriate domain).
	\begin{enumerate}[label = \textbullet]
		\item \((\mathbb{Z}, d)\) is a metric space.
		\item \((\mathbb{Q}, d)\) is a metric space.
		\item \((\mathbb{R}, d)\) is a metric space.
		\item \((\mathbb{C}, d)\) is a metric space.
	\end{enumerate}
\end{exmp}

Every normed vector space \(W\) is naturally also a metric space, by considering the metric \(d: W \times W \to \mathbb{R}\) defined by \(d(v, u) = ||v - u||\).

\begin{exmp}{\(\mathbb{R}^n\) is a metric space}{}
	Define \(d \colon \mathbb{R}^n \times \mathbb{R}^n \to \mathbb{R}\) by \(d(\mathbf x, \mathbf y) = ||\mathbf x - \mathbf y||\). \((\mathbb{R}^n, d)\) is a metric space.
\end{exmp}

There are metric spaces which are not normed vector spaces, but they are out of the scope of this course.

From now on in these notes, whenever you read ``\(X\) is a metric space'' or ``\(X\) is a normed vector space,'' it is useful to think about the prototypical examples of \(X = \mathbb{R}\), \(X = \mathbb{C}\), or \(X = \mathbb{R}^2\). 

\chapter{Limits}

\section{Sequences}

\begin{defn}{Limit of a sequence}{limitsequence}
	Let \(X\) be a metric space.
	We say a sequence \((a_n) = a_1, a_2, a_3, \dots\), where \(a_i \in X\), \emph{converges to \(a \in X\)} if, for every \(\epsilon > 0\), there exists \(N \in \mathbb{N}\) such that \(d(a_n, a) < \epsilon\) for all \(n \geq N\).

	If this definition holds for some \(a\), we write \(\lim_{n\to\infty} a_n = a\) or \(a_n \to a\).

	If this definition does not hold for any \(a\), we say \(\lim_{n\to\infty} a_n\) does not exist, or that the sequence diverges.
\end{defn}

\begin{prop}{The limit, if it exists, is unique}{}
	If  \(\lim_{n\to\infty} a_n = a\) and  \(\lim_{n\to\infty} a_n = a'\), then \(a = a'\).
\end{prop}

\begin{dem}{}{}
	For all \(\epsilon > 0\), there exists \(N_\epsilon \in \mathbb{N}\) such that \( d(a_n, a) <\epsilon\) for all  \(n \geq N_\epsilon\).
	For all \(\epsilon > 0\), there exists \(M_\epsilon \in \mathbb{N}\) such that \( d(a_n - a') <\epsilon\) for all  \(n \geq M_\epsilon\).

	Therefore, for all \(\epsilon > 0\), there exists \(L_\epsilon \in \mathbb{N}\), namely \(\max\{N_\epsilon, M_\epsilon\}\), such that  \(d(a_n, a) < \epsilon\) and  \(d(a_n, a') < 0\) for all  \(n \geq L_\epsilon\). Triangle inequality implies that \(d(a, a') < 2\epsilon\) for all \(\epsilon > 0\); thus \(d(a, a') = 0\), and consequently \(a = a'\).
\end{dem}

\begin{exmp}{}{}
	We claim that \(\lim_{n\to\infty} \frac{1}{n^2} = 0\).

	This is true because, given \(\epsilon > 0\), we can choose \(N\) be a natural number larger than \(\sqrt{\frac{1}{\epsilon}}\). Then, for all \(n \geq N\), we have \[
		\epsilon > \frac{1}{N^2} > \frac{1}{n^2} = \left|\frac{1}{n^2} - 0\right|.
	\]
\end{exmp}

\begin{exmp}{The limit does not exist}{}
	We claim that \(\lim_{n\to\infty} (-1)^n\) does not exist.  

	Suppose it does exist, namely \(a\). Then, consider \(\epsilon = \frac{1}{2} \max\{|a - 1|, |a + 1|\}\). Not both \(|a - 1|\) and  \(|a + 1|\) can be zero, so  \(\epsilon > 0\). However, since \(\lim_{n\to\infty} = a\), for that \(\epsilon\), it must hold that there exists \(N \in \mathbb{N}\) so that for all \(n \geq N\), \(|a - (-1)^n| < \epsilon\).

	In particular, note that plugging in \(n \mapsto N\) and \(n\mapsto N+1\) imply that \(|a - 1| < \epsilon\) and \(|a + 1| < \epsilon\); which is a contradiction given our choice of \(\epsilon\).
\end{exmp}

\begin{exmp}{}{}
	We claim that \(\lim_{n\to\infty} \frac{2n+1}{n+3} = 2\). 
	Note that we can rewrite \(\frac{2n+1}{n+3} = 2 - \frac{5}{n+3}\). For any \(\epsilon\), there exists \(N \in \mathbb{N}\) such that \(N > \frac{5}{\epsilon}\). Therefore, for all \(n \geq N\), it holds that \[
		\left|\left(2 - \frac{5}{n+3}\right) - 2\right| = \frac{5}{n+3} < \frac{5}{N} < \epsilon\text{,}
	\]
	and our claim follows.
\end{exmp}
