\chapter{What are the real numbers?}
\lecture{1}{2021-08-30}{}
\section{Defining the real numbers: an axiomatic aproach}

The main idea is to derive \(\mathbb{R}\) from \(\mathbb{Q}\). We will layout some properties that \(\mathbb{Q}\) has that we also want \(\mathbb{R}\) to have; and then add an additional property that will distinguish \(\mathbb{Q}\) from \(\mathbb{R}\).

First, \(\mathbb{Q}\) is a field, and we also want \(\mathbb{R}\) to be a field.

\begin{defn}{Field Axioms}{field}
	A set \(F\) is a \emph{field} if there exist two operations --- addition and multiplication --- that satisfy the following list of conditions:
	\begin{enumerate}
		\item (Commutativity) \(x + y = y + x\) and \(xy = yx\) for all \(x, y \in F\).
		\item (Associativity) \((x+y)+z = x+(y+z)\) and \((xy)z=x(yz)\) for all \(x, y, z \in F\).
		\item (Identities) There exist two special elements, denoted by \(0\) and \(1\), such that \(x + 0 = x\) and \(x1 = x\) for all \(x \in F\).
		\item (Inverses) Given \(x \in F\), there exists an element \(-x \in F\) such that \(x + (-x) = (-x) + x = 0\). If \(x \neq 0\), there exits an element \(x^{-1}\) such that \(xx^{-1} = x^{-1}x = 1\).
		\item (Distributivity) \(x(y+z) = xy + xz\) for all \(x, y, z \in F\).
	\end{enumerate}
\end{defn}

Being a field is not restrictive enough, since it allows for finite fields, such as \(\mathbb{Z}/p\mathbb{Z}\), or complex numbers \(\mathbb{C}\). Another feature of \(\mathbb{Q}\) (and a desired feature of \(\mathbb{R}\)) is order.

\begin{defn}{Ordering}{ordering}
	An \emph{ordering} on a set \(F\) is a relation, represented by \(\leq\), with the following properties:
	\begin{enumerate}
		\item \(x \leq y\) or \(y \leq x\), for all \(x, y \in F\).
		\item If \(x \leq y\) and \(y \leq x\), then \(x = y\).
		\item If \(x \leq y\) and \(y \leq z\), then \(x \leq z\).
	\end{enumerate}

	We define \(x < y\) as equivalent to \(x \leq y\) and \(x \neq y\).
	We define \(y \geq x\) as equivalent to \(x \leq y\).
	We define \(y > x\) as equivalent to \(x < y\).

	Additionally, a field \(F\) is called an \emph{ordered field} if \(F\) is endowed with an ordering \(\leq\) that satisfies
	\begin{enumerate}[resume]
		\item If \(y \leq z\), then \(x + y \leq x + z\).
		\item If \(x \geq 0\) and \(y \geq 0\), then \(xy \geq 0\).
	\end{enumerate}
\end{defn}

Now, we need to add a feature that distinguishes \(\mathbb{Q}\) and our desired \(\mathbb{R}\).
Intuitively, ``\(\mathbb{Q}\) has holes'', meaning that one can build a sequence in \(\mathbb{Q}\) that approaches a limit that is not in \(\mathbb{Q}\); on the other hand, ``\(\mathbb{R}\) has no holes'', meaning that any sequence in \(\mathbb{R}\) that converges can only converge to a limit that is in \(\mathbb{R}\).

