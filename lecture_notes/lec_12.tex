\lecture{12}{2021-09-27}{}

\begin{thm}{Squeeze Theorem}{squeeze}
	Let \((x_n)\), \((y_n)\), and \((z_n)\) be sequences of real numbers.
	If $x_n \le y_n \le z_n$ for all $n \in \mathbb{N}$, and if $\lim_{n\to\infty} x_n = \lim_{n\to\infty} z_n = L$, then $\lim_{n\to\infty} y_n = L$.  
\end{thm}

\begin{dem}{}{}
	For all \(n \in \mathbb{N}\), since \(x_n \leq y_n \leq z_n\), \( |z_n - x_n| = |z_n - y_n| + |y_n - x_n| \), which implies
	\begin{equation} \label{eqn:zn-xn>zn-yn}
		|z_n - x_n| \geq |y_n - x_n|.
	\end{equation}

	Theorem~\ref{thm:manipulationlimits} implies that \(\lim_{n\to\infty} (z_n - x_n) = \lim_{n\to\infty} z_n - \lim_{n\to\infty} x_n = 0\).

	Let \(\epsilon > 0\). Therefore, since \((z_n - x_n) \to 0\), there exists \(N\) such that \(|z_n - x_n| < \epsilon\) for all \(n \geq N\).
	Equation~(\ref{eqn:zn-xn>zn-yn}) implies that, for the same choice of \(N\), it holds that \(|y_n - x_n| < \epsilon\) for all \(n \geq N\). Therefore,  \((y_n - x_n) \to 0\).
	Since \((x_n) \to L\) and \((y_n - x_n) \to 0\), theorem~\ref{thm:manipulationlimits} implies \((y_n) \to L\).
\end{dem}

\begin{exmp}{}{}
	We claim that \(\lim_{n\to\infty} \sqrt{n^2 + 4n} - n = 2\).

	A good intuition for that to be true is that \(\sqrt{n^2 + 4n} - n \approx \sqrt{n^2 + 4n + 4} - n = 2\).

	Formally, \begin{align*}
		\sqrt{n^2 + 4n} - n &= \frac{(n^2 + 4n) - n^2}{\sqrt{n^2 + 4n} + n^2}\\
							&= \frac{4}{\sqrt{1 + 4/n} + 1} \to 2.
	\end{align*}
\end{exmp}

\begin{thm}{Bolzano-Weierstrass Theorem}{bw}
	Every bounded sequence of real numbers has a convergent subsequence.
\end{thm}

\begin{dem}{}{}
	Since \((a_n)\) is bounded, there exists \(M\) such that \(a_n \leq M\) for all \(n\). Let \(I_1 = [-M, M]\). Note that infinitely many terms of \((a_n)\) are in \(I_1\).

	Suppose \(I_k = [a_k, b_k]\) contains infinitely many terms of \((a_n)\). Define \(I_{k+1}\) as either \([a_k, \frac{a_k + b_k}{2}]\) or  \([\frac{a_k+b_k}{2}, b_k]\) such that \(I_{k+1}\) contais infinitely many terms of \((a_n)\).

	\nameref{thm:nestedintervalproperty} implies that there exists \(x \in I_j\) for all \(j\).

	Let \(n_1 = 1\), so that \(a_{n_1} \in I_1\). Define \(n_{i+1} > n_i\), so that \(a_{n_{i+1}} \in I_{i+1}\); which is possible since \(I_{n+1}\) has infinitely many terms.

	For each \(j\), both \(a_{n_j}\) and \(x\) are in \(I_j\). Since the width of \(I_j\) is \(2M/2^{j-1}\), we conclude  \[
		-\frac{2M}{2^{j-1}} + x \leq a_{n_j}  \leq \frac{2M}{2^{j-1}} + x,
	\]
	thus the \nameref{thm:squeeze} implies \((a_{n_j}) \to x\).
\end{dem}

\begin{defn}{Cauchy sequence}{}	
	Let \(X\) be a metric space. 
	A sequence of elements in \(X\) is \emph{Cauchy} if, for all \(\epsilon > 0\), there exists \(N\) so that \(d(a_m, a_n) < \epsilon\) for all \(m, n \geq N\).
\end{defn}

\begin{exmp}{}{}
	We claim that the sequence \(a_n = \frac{(-1)^n}{n}\) is Cauchy.

	Let \(\epsilon > 0\). Choose \(N\) larger than \(\frac{1}{2\epsilon}\).

	Then, for all \(n, m \geq N\), it holds that
	\begin{align*}
		\left|\frac{(-1)^n}{n} - \frac{(-1)^m}{m}\right| &= \left|\frac{1}{n} \pm \frac{1}{m}\right| \\
		&\leq \frac{1}{n} + \frac{1}{m} \\
		&\leq \frac{2}{N} \\
		&< \epsilon.
	\end{align*}
\end{exmp}

\begin{prop}{}{}
	Every convergent sequence is Cauchy.
\end{prop}

\begin{dem}{}{}
	Let \(\epsilon > 0\). Since \((a_n) \to L\), there exists \(N\) so that  \[
		d(a_n, L) < \frac{\epsilon}{2}
	\]
	for all \(n \geq N\). Therefore, using the triangle inequality, \[
		d(a_n, a_m) \leq d(a_n, L) + d(L, a_m) < \epsilon
	\]
	for all \(n, m \geq N\); thus the sequence is Cauchy.
\end{dem}
