\lecture{13}{2021-09-29}{}

\begin{prop}{}{cauchybounded}
	Let \(W\) be a normed vector space.
	Every Cauchy sequence of elements in \(W\) is a bounded sequence.
\end{prop}

\begin{dem}{}{}
	Let \(\epsilon = 1\). There exist \(N\) so that \(|a_m - a_n| < 1\) for all \(m, n \geq N\). This implies that \(|a_m - a_N| < 1\) for all \(m \geq N\), and consequently, by triangle inequality, \(|a_m| = |a_m - 0| \leq |a_m - a_N| + |a_N - 0| < 1 + |a_N|\) for all \(m \geq N\).

	Therefore, if we set \[
		M = \max\{|a_1|, |a_2|, \dots, |a_{N-1}|, |a_N| + 1\},
	\]
	we conclude \(|a_m| < M\) for all \(m\).
\end{dem}

\begin{prop}{}{subsequencecauchyconverges}
	Let \(X\) be a metric space. Let \((a_n)\) be a sequence of elements of \(X\).
	If \((a_n)\) is Cauchy, and if some subsequence of \((a_n)\) converges to some limit \(a \in X\), then the whole sequence \((a_n)\) converges to \(a \in X\).
\end{prop}

\begin{dem}{}{}
	Let \(\epsilon > 0\). Let \((a_{k_i})\) be such sequence that converges to \(a\).
	Thus, there exists \(N\) so that \[
		d(a_{k_n}, a) < \epsilon/2
	\]
	for all \(n > N\).

	Also, since \((a_n)\) is Cauchy, there exists \(M\) so that \[
		d(a_m, a_n) < \epsilon/2
	\] for all \(m, n \geq M\).
	In particular, by setting \(m = k_n \geq n\), we conclude \[
		d(a_{k_n}, a_n) < \epsilon/2
	\] for all \(n \geq M\).

	Therefore, for all \(n \geq \max\{N, M\}\), it holds that \[
		d(a_n, a) \leq d(a_n, a_{k_n}) + d(a_{k_n}, a) < \epsilon;
	\] in other words, \((a_n)\) converges to \(a\).
\end{dem}

\begin{thm}{}{cauchyconvergentreal}
	Every Cauchy sequence of real numbers is convergent.
\end{thm}

\begin{dem}{}{}
	Let \((a_n)\) be a Cauchy sequence o

	\textcolor{red}{To be finished.}
\end{dem}

