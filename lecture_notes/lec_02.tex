\lecture{2}{2021-09-01}{Sup, inf, abs, completeness}

\subsection{Bounds}

Now, we need to add a feature that distinguishes \(\mathbb{Q}\) and our desired \(\mathbb{R}\).
Intuitively, ``\(\mathbb{Q}\) has holes'', meaning that one can build a sequence in \(\mathbb{Q}\) that approaches a limit that is not in \(\mathbb{Q}\); on the other hand, ``\(\mathbb{R}\) has no holes'', meaning that any sequence in \(\mathbb{R}\) that converges can only converge to a limit that is in \(\mathbb{R}\).

\begin{defn}{Upper bound}{upperbound}
	If \(F\) is an ordered field, and \(A \subset F\), then we say that some \(b \in F\) is an \emph{upper bound of \(A\) } if \(a \leq b\) for all \(a \in A\).

	If a set \(A\) has an upper bound, we say that \(A\) is \emph{bounded above}.
\end{defn}

\begin{defn}{Supremum}{supremum}
	If \(F\) is an ordered field, and \(A \subset F\), we say \(s \in F\) is the \emph{least upper bound of \(A\)}, or \emph{supremum of \(A\)}, denoted by \(\sup(A)\), if:
	\begin{enumerate}
		\item \(s\) is an upper bound of \(A\), and
		\item if \(b\) is any upper bound of \(A\), then \(s \leq b\).
	\end{enumerate}
\end{defn}

\begin{prop}{The supremum, if it exists, is unique}{supisunique}
	If \(s\) and \(s'\) are both supremum of \(A\), then \(s = s'\).
\end{prop}

\begin{exmp}{}{}
	Let \(F = \mathbb{Q}\) and \(A = \{0, -1, -\frac{1}{2}, -\frac{1}{3}, -\frac{1}{4}, \dots\}\). Then,  \(\sup(A) = 0\).
\end{exmp}

Analogously, we can define lower bounds and least upper bounds.

\begin{defn}{Lower bound}{lowerbound}
	If \(F\) is an ordered field, and \(A \subset F\), then we say that some \(b \in F\) is a \emph{lower bound of \(A\) } if \(a \geq b\) for all \(a \in A\).

	If a set \(A\) has a lower bound, we say that \(A\) is \emph{bounded below}.
\end{defn}

\begin{defn}{Infimum}{infimum}
	If \(F\) is an ordered field, and \(A \subset F\), we say \(s \in F\) is the \emph{greatest lower bound of \(A\)}, or \emph{infimum of \(A\)}, denoted by \(\inf(A)\), if:
	\begin{enumerate}
		\item \(s\) is a lower bound of \(A\), and
		\item if \(b\) is any lower bound of \(A\), then \(s \geq b\).
	\end{enumerate}
\end{defn}

\subsection{Completeness}

\begin{defn}{Completeness}{completeness}
	Given \(F\) an ordered field, we say \(F\) is \emph{complete} if, for any subset \(A \subset F\) bounded above and nonempty, the supremum of \(A\) exists\footnote{and is an element of \(F\), as the definition requires.}.
\end{defn}

\begin{thm}{Unique complete ordered field}{uniqueorderedfield}
	There exists a unique complete ordered field, up to isomorphism. 
\end{thm}

The proof of this theorem is beyond the scope of this course. One can show the existence of such a field by creating a field of Dedekind cuts. A Dedekind cut is a subset \(C \subset \mathbb{Q}\) such that, if \(c \in C\), then all rational numbers \(x < c\) also are in \(C\); and that \(C\) has no supremum in \(\mathbb{Q}\). Addition can be defined using set addition. Multiplication is harder to define, since it is needed a separation between ``non-negative'' and ``negative'' numbers. Ordering can be defined using subsets. Finally, one has to prove all the axioms (field axioms, ordering axioms, and the axiom of completeness).

\begin{defn}{Real numbers}{realnumbers}
	The set of real numbers, denoted by \(\mathbb{R}\), is the complete ordered field.
\end{defn}

\begin{que}{}{}
	If \(\mathbb{R}\) is defined in such a axiomatic way, how can we say that \(\mathbb{Z} \subset \mathbb{R}\) and \(\mathbb{Q} \subset \mathbb{R}\)?
\end{que}

Recall that the statement \(\mathbb{Z} \subset \mathbb{Q}\) is also a strange statement. A rational number is actually a pair of integers; how can a single integer be also a pair of integers? To be fair, in a set-theoretical sense, it is in fact untrue that \(\mathbb{Z} \subset \mathbb{Q}\). However, the set of rational numbers of the form \(\frac{n}{1}\) has the same structure (with respect to multiplication and addition) as the set of integers. Therefore, when we say ``\(\mathbb{Z} \subset \mathbb{Q}\)'', we actually mean that there exists a subset of \(\mathbb{Q}\) that is isomorphic to \(\mathbb{Z}\). This difference is usually not interesting for us, when studying Analysis. In the rare cases where this kind of difference is relevant, we say sentences like ``there is a copy of \(\mathbb{Z}\) in \(\mathbb{Q}\).''

In a similar fashion, the additive group generated by the identity of \(\mathbb{R}\)\footnote{In other words, the smallest additive group that contains the identity of \(\mathbb{R}\).} is isomorphic to \(\mathbb{Z}\); as well as the field generated by the identity of \(\mathbb{R}\)\footnote{In other words, the smallest field that contains the identity of \(\mathbb{R}\).} is isomorphic to \(\mathbb{Q}\). Therefore, there are copies of \(\mathbb{Z}\) and \(\mathbb{Q}\) in \(\mathbb{R}\), or, more informally, \(\mathbb{Z} \subset \mathbb{R}\) and \(\mathbb{Q} \subset \mathbb{R}\).

\section{Appendix: Absolute value}

Before we dig more deeply into the idea of a supremum, consider this definition that comes just from the structure of an ordered field.

\begin{defn}{Absolute value}{absolutevalue}
	If \(F\) is an ordered field, and \(x \in F\), let \[
		|x| =
		\begin{cases}
			x \text{\ if\ } x \geq 0\text{,} \\
			-x \text{\ if\ } x < 0\text{.}
		\end{cases}
	\]
\end{defn}

\begin{thm}{}{}
	If \(F\) is an ordered field, and \(x \in F\), then \(|x| \geq 0\).
\end{thm}

\begin{dem}{}{}
	If \(x \geq 0\), then \(|x| = x \geq 0\).
	If \(x \leq 0\), then \(0 = x + (-x) \leq 0 + (-x) = -x = |x|\).
\end{dem}

\begin{thm}{}{|-x|=|x|}
	If \(F\) is an ordered field, and \(x \in F\), then \(|-x| = |x|\).
\end{thm}

\begin{dem}{}{}
	If \(x \geq 0\), then \(0 = x + (-x) \geq 0 + (-x) = -x\), therefore \(|-x| = -(-x) = x = |x|\).
	If \(x \leq 0\), then \(0 = x + (-x) \leq 0 + (-x) = -x\), therefore \(|-x| = -x = |x|\).
\end{dem}

\begin{thm}{}{}
	If \(F\) is an ordered field, and \(x, y \in F\), then \(|xy| = |x||y|\).
\end{thm}

\begin{dem}{}{}
	If \(x \geq 0\) and \(y \geq 0\), then \(xy \geq 0\) and \(|xy| = xy = |x||y|\).

	If \(x \geq 0\) and \(y \leq 0\), then \(0 = y + (-y) \leq 0 + (-y) = -y\). So we apply the previous case with \(x\) and \(-y\) and also Theorem \ref{thm:|-x|=|x|} to obtain \(|xy| = |-xy| = |x(-y)| = |x||-y| = |x||y|\).

	If \(x \leq 0\) and \(y \geq 0\), then \(0 = x + (-x) \leq 0 + (-x) = -x\). So we apply the first case with \(-x\) and \(y\) and also Theorem \ref{thm:|-x|=|x|} to obtain \(|xy| = |-xy| = |(-x)y| = |-x||y| = |x||y|\).

	If \(x \leq 0\) and \(y \leq 0\), then \(0 = x + (-x) \leq 0 + (-x) = -x\) and \(0 = y + (-y) \leq 0 + (-y) = -y\). So we apply the first case with \(-x\) and \(-y\) and also Theorem \ref{thm:|-x|=|x|} to obtain \(|xy| = |(-x)(-y)| = |-x||-y| = |x||y|\).
\end{dem}

\begin{thm}{Triangle inequality}{triangleinequality}
	If \(F\) is an ordered field, and \(x, y \in F\), then  \[ |x + y| \leq |x| + |y|.\]
\end{thm}

\begin{dem}{}{}
	If \(x \geq 0\), then \(|x| = x\). If \(x \leq 0\), then \(x \leq 0 = x + (-x) \leq 0 + (-x) = x\), so \(|x| = -x \geq x\). In either case, \(|x| \geq x\).

	Thus,  \(|x| + |y| \geq x + y\)	and \(|x| + |y| = |-x| + |-y| \geq - x - y = -(x + y)\). Since  \(|x + y| = x + y\) or  \(|x + y| = -(x+y)\), in either case, \(|x| + |y| \geq |x + y|\).
\end{dem}

\begin{thm}{Reverse triangle inequality}{reversetriangleinequality}
	If \(F\) is an ordered field, and \(x, y \in F\), then \[
		|x - y| \geq ||x| - |y||.
	\]
\end{thm}

\begin{dem}{}{}
	Triangle inequality implies that \(|x| = |(x - y) + y| \leq |x - y| + |y|\) and  \(|y| = |(y - x) + x| \leq |y - x| + |x|\). Equivalently, we have \(|x - y| \geq |x| - |y|\) and \(|x - y| \geq |y| - |x|\); consequently, \(|x - y| \geq ||x| - |y||\).
\end{dem}

