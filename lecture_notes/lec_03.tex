\chapter{Getting to know the Real Numbers}

\lecture{3}{2021-09-03}{e-sup, sup sum of sets}

\begin{thm}{\(\epsilon\)-sup Theorem}{e-sup}
	Given \(A \subset \mathbb{R}\) nonempty and bounded above, and given \(s\) an upper bound of \(A\), then \(s = \sup(A)\) if, and only if, for all \(\epsilon > 0\), there exists \(a \in A\) such that \(a > s - \epsilon\).
\end{thm}

\begin{dem}{}{}
	Supppose \(s = \sup(A)\). Then, \(s - \epsilon\) is not an upper bound of \(A\). Therefore, there exists \(a \in A\) such that \(a > s - \epsilon\).

	Suppose \(s \neq \sup(A)\). Then, there exists an upper bound of \(A\) that is smaller than \(s\), say \(s - \delta\). Then, it follows that, for \(\epsilon = \delta / 2\), there is no \(a \in A\) such that \(a > s - \delta/2\), because  \(a \leq s - \delta < s - \delta/2\) for all \(a \in A\).
\end{dem}

\begin{defn}{Sum of Sets}{sumofsets}
	Given \(A, B \subset \mathbb{R}\), we define their sum as \[
		A + B = \{a + b : a \in A, b \in B\}
	\]
\end{defn}

\begin{thm}{Supremum of Sum of Sets}{}
	If \(A, B \subset \mathbb{R}\) are both nonempty and bounded above, then \[
		\sup(A+B) = \sup(A) + \sup(B).
	\]
\end{thm}

\begin{dem}{}{}
	Since \(\sup(A)\) is an upper bound of \(A\), it holds that \(a \leq \sup(A)\) for all \(a \in A\).
	Since \(\sup(B)\) is an upper bound of \(B\), it holds that \(b \leq \sup(B)\) for all \(b \in B\).
	Therefore, \(a + b \leq \sup(A) + \sup(B)\) for all \(a \in A\) and \(b \in B\), i.e., \(x \leq \sup(A) + \sup(B)\) for all \(x \in A + B\); thus, \(\sup(A) + \sup(B)\) is an upper bound of \(A + B\).

	Let \(\epsilon > 0\) be any positive real number.
	\nameref{thm:e-sup} implies that there exists \(a \in A\) such that \(a > \sup(A) - \epsilon/2\).
	\nameref{thm:e-sup} also implies that there exists \(b \in B\) such that \(b > \sup(B) - \epsilon/2\).
	Therefore, there exist \(a \in A\) and \(b \in B\) such that \(a + b > \sup(A) + \sup(B) - \epsilon\); thus, there exists \(x \in X\) such that \(x > \sup(A) + \sup(B) - \epsilon\).
	Finally, by \nameref{thm:e-sup}, \(\sup(A+B) = \sup(A) + \sup(B)\).
\end{dem}
