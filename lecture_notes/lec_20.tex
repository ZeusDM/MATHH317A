\lecture{20}{2021-10-22}{}

\begin{defn}{Closure}{closure}
	The closure of \(A\), denoted by \(\overline A\), is the union of \(A\) with the set of its limit points.
\end{defn}

\begin{defn}{Interior}{interior}
	The interior of \(A\), denoted by \(\mathring A\), is the set  \[
		\mathring A = \{x \in A : \exists \epsilon>0,  V\epsilon(x) \subset A \}
	\]
\end{defn}

\begin{defn}{Boundary}{boundary}
	The boundary of \(A\), denoted by \(\partial A\), is \(\overline A - \mathring A\).
\end{defn}

\begin{prop}{}{}
	\(\overline A\) is closed and, for any closed \(C\) containing \(A\), it also holds that \(C\) contains \(\overline A\).

	In this sense, \(\overline A\) is the smallest closed set that contains \(A\).
\end{prop}

\begin{prop}{}{}
	\(\mathring A\) is open and, for any open set \(O \subset A\), it also holds taht \(O \mathring A\).

	In this sense, \(\mathring A\) is the largest open set contained in \(A\).
\end{prop}

\begin{thm}{Sequence Interpretation of Limit Point}{}
	Given a set \(A\) in a metric space \(X\), \(w \in W\) is a limit point of \(A\) if, and only if, there exists a sequence \((v_n)\) in \(A\) so that \(v_n \neq w\) for all \(n\) and \(\lim_{n\to\infty} v_n = w\).
\end{thm}

\begin{thm}{Sequence Interpretation of a Closed Set}{}
	Given a set \(A\) in a metric space \(X\), \(A\) is closed if, and only if, for any sequence \((v_n)\) in \(A\) that has limit \(w\) in \(W\), it holds that \(w \in A\).
\end{thm}
