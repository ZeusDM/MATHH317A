\lecture{22}{2021-10-27}{}

\begin{defn}{Open Cover}{}
	Given \(A \subset X\), and a collection \(\{O_\lambda\}\) of open sets in  \(X\), we call \(\{O_X\}\) an \emph{open cover of \(A\)}, if \(A \subset \bigcup_\lambda O_\lambda\).

	A \emph{open subcover} of a given cover \(\{O_\lambda\}\) of \(A\) is a subset of \(\{O_\lambda\}\) that still covers \(A\).
\end{defn}

\begin{defn}{Compactness}{compact}
	We say taht \(A \subset X\) is \emph{compact} if every open cover of \(A\) admits a finite subcover.
\end{defn}

\begin{exmp}{}{}
	I claim that the set \([0, 1)\) is not compact. Consider the open cover \[
		(-1/2, 1/2), (-2/3, 2/3), (-3/4, 3/4), \dots.
	\]
	Their union is \((-1, 1)\), which covers \([0, 1)\). However, any finite subcover will have union of the form \((-(n-1)/n, (n-1)/n)\), which does not cover \([0,1)\).
\end{exmp}

\begin{defn}{Sequentially Compactness}{sequentaillycompact}
	We say that \(A \subset X\) is \emph{sequentially compact} if every sequence in \(A\) has a subsequence that converges to an element of \(A\).
\end{defn}

\begin{exmp}{}{}
	I claim that the set \(\mathbb{N}\) is not sequentially compact. Consider the sequence \[
		1, 2, 3, 4, 5, \dots.
	\]
	No subsequence of this sequence converges.

	I also claim that the set \(\mathbb{N}\) is not compact. Consider the open cover \[
		(1 - \tfrac{1}{2}, 1+\tfrac{1}{2}), (2 - \tfrac{1}{2}, 2+\tfrac{1}{2}), (3-\tfrac{1}{2},3+\tfrac{1}{2}), \dots.
	\]
	No finite subset of this open cover also covers \(A\).
\end{exmp}

\begin{thm}{Heine-Borel}{heine-borel}
	If \(A \subset \mathbb{R}^n\), the following statements are equivalent:
	\begin{enumerate}
		\item \(A\) is compact;
		\item \(A\) is sequentially compact;
		\item \(A\) is closed and bounded.
	\end{enumerate}
\end{thm}
