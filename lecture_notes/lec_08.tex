\lecture{8}{2021-09-17}{Prop of limits}

\begin{exmp}{}{}
	We claim that \(\lim_{n\to\infty} \frac{2n^2}{5n^3 - 7} = 0\).

	This is true because, given  \(\epsilon > 0\), we can choose \(N\) to be a natural number larger than \(\frac{1}{\epsilon}\) and larger than \(2\). Then, for all  \(n \geq N\), we have \begin{align*}
		\epsilon > \frac{1}{N} > \frac{1}{n} > \frac{2n^2}{4n^3} > \frac{2n^2}{4n^3 + (n^3 - 7)} = \left| \frac{2n^2}{5n^3 - 7} - 0 \right|
	\end{align*}
\end{exmp}

\begin{thm}{}{}
	Given a sequence \((v_n)\) in a metric space \(X\), \[
		\lim_{n\to\infty} v_n = L \text{ if, and only if, } \lim_{n\to\infty} d(v_n, L) = 0.
	\]
\end{thm}

\begin{prop}{}{}
	Given a sequence \(\vec v_n = (x_{1, n}, x_{2, n}, \dots, x_{k, n})\) in  \(\mathbb{R}^k\), \[
		\lim_{n\to\infty} \vec v_n = (L_1, L_2, \dots, L_k) \text{ if, and only if, } \lim_{n\to\infty} x_{i, n} = L_i \text{ for all \(1 \leq i \leq k\)}.
	\]
\end{prop}


\begin{thm}{Algebraic Manipulation of Limits}{manipulationlimits}
	Let \(W\) be a normed vector space over \(F\).

	Suppose that \(\lim_{n\to\infty} a_n = a\), \(\lim_{n\to\infty} b_n = b\) are elements of \(W\) and \(c, d \in F\).
	Then, 
	\begin{enumerate}
		\item \(\lim_{n\to\infty} (ca_n + db_n) = ca + db\)
	\end{enumerate}

	If \(W\) is a field, then,
	\begin{enumerate}[resume]
		\item \(\lim_{n\to\infty} a_nb_n = ab\)
		\item \(\lim_{n\to\infty} (1/a_n) = 1/a\) if the \(a_n \neq 0\) for all  \(n\) and \(a \neq 0\).
		\item \(\lim_{n\to\infty} (a_n/b_n) = a/b\) if the \(b_n \neq 0\) for all  \(n\) and \(b \neq 0\).
	\end{enumerate} 
\end{thm}

\begin{dem}{}{}
	\begin{enumerate}
		\item Let \(\epsilon > 0\).
			Since \(\lim_{n\to\infty} a_n = a\), there exist \(N\) such that \[
				\left|a_n - a\right| < \frac{\epsilon}{2|c|}
			\] 
			for all \(n \geq N\).
			Similarly, there exists \(M\) such that \[
				\left| b_n - b \right| < \frac{\epsilon}{2|d|}
			\]
			for all \(n \geq M\).
			Therefore, for all \(n \geq \max\{N, M\}\), it holds that
			\begin{align*}
				\left|(ca_n + db_n) - (ca + db)\right| &= \left|(ca_n - ca) + (db_n - db)\right| \\
													   &\leq |ca_n - ca| + |db_n - db| \\
													   &\leq |c||a_n-a| + |d||b_n-b| \\
													   &< \epsilon,
			\end{align*}
			thus, \(\lim_{n\to\infty} ca_n + db_n\).
		\item Let \(\epsilon > 0\).
			Since \(\lim_{n\to\infty} a_n = a\), there exist \(N\) such that \[
				|a_n - a| < 1
			\] 
			for all \(n \geq N\); therefore, \(|a_n| < |a| + 1\) for all \(n \geq N\).

			Since \(\lim_{n\to\infty} a_n = a\), there exist \(M\) such that \[
				|a_n - a| < \frac{\epsilon}{|b|}
			\] 
			for all \(n \geq M\).
			Similarly, there exist \(O\) such that \[
				|b_n - b| < \frac{\epsilon}{2 (|a| + 1)}
			\] 
			for all \(n \geq O\).
			Therefore, for all  \(n \geq \max\{N, M, O\}\), it holds that
			\begin{align*}
				|a_nb_n - ab| &= |a_nb_n - a_nb + a_nb - ab| \\
							  &= |a_n(b_n - b) + b(a_n - a)| \\
							  &\leq |a_n(b_n - b)| + |b(a_n - a)| \\
							  &< \epsilon.
			\end{align*}
			Therefore, \(\lim_{n\to\infty} a_nb_n = ab\).

		\item Triangle inequality implies that \(|a| \leq |a - a_n| + |a_n|\); thus \(|a_n| \geq |a| - |a - a_n|\).
			Since \(\lim_{n\to\infty} a_n = a\), there exist \(N\) such that \[
				|a_n - a| < \frac{|a|}{2}.
			\]
			for all \(n \geq N\).  Therefore, \(|a_n| > |a| - \frac{|a|}{2} = \frac{|a|}{2}\), and consequently, \(\frac{2}{|a|} > \left|\frac{1}{a_n}\right|\) for all \(n \geq N\).

			Let \(\epsilon > 0\). Since \(\lim_{n\to\infty} a_n = a\), there exists \(M\) so that \[
				|a_n - a| < \frac{\epsilon |a|^2}{2}.
			\]
			Then, for all \(n \geq \max\{N, M\}\), it holds that
			\begin{align*}
				\left|\frac{1}{a_n} - \frac{1}{a}\right| &= |a - a_n|\cdot \left|\frac{1}{a}\right| \cdot \left|\frac{1}{a_n}\right|\\
				&< \frac{\epsilon |a|^2}{2} \cdot \frac{1}{|a|} \cdot \frac{2}{|a|} \\
				&< \epsilon.
			\end{align*}

			Therefore, \(\lim_{n\to\infty} \frac{1}{a_n} = \frac{1}{a}\).
		\item Using \textbf{ii} and \textbf{iii}, we have 
			\begin{align*}
				\lim_{n\to\infty} \frac{a_n}{b_n} &= \lim_{n\to\infty} \left( a_n \frac{1}{b_n} \right) \\
												  &= \left(\lim_{n\to\infty} a_n\right) \left(\lim_{n\to\infty}\frac{1}{b_n} \right) \\
												  &= a\cdot \frac{1}{b} = \frac{a}{b}. 
			\end{align*}
	\end{enumerate}
\end{dem}

\begin{exmp}{}{}
	Since \(\lim_{n\to\infty} (1 + 1/n) = \lim_{n\to\infty} 1 + \lim_{n\to\infty} (1/n) = 1 + 0 = 1\)
	and \(\lim_{n\to\infty} (1 + 1/n^2) = \lim_{n\to\infty} 1 + \lim_{n\to\infty} (1/n^2) = 1 + 0 = 1\), we can conlude that
	\begin{align*}
		\lim_{n\to\infty} \frac{n^2 + n}{n^2 + 1} &= \lim_{n\to\infty} \frac{1 + 1/n}{1 + 1/n^2} \\
												  &= \frac{\lim_{n\to\infty} 1 + 1/n}{\lim_{n\to\infty} 1 + 1/n^2} \\
												  &= \frac{1}{1} = 1.
	\end{align*}
\end{exmp}
