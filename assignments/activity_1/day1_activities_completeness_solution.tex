\documentclass[11pt]{amsart}
\usepackage{graphicx}
\usepackage{amssymb, enumitem, fullpage, parskip}
\pagestyle{empty}

\newcommand{\floor}[1]{\left\lfloor #1 \right\rfloor}
\newcommand{\ceil}[1]{\left\lceil #1 \right\rceil}

\title{Math 317 Fall 2021: In-Class Activity for Mon. Aug. 30 (Completeness)}
\author{Guilherme Zeus Dantas e Moura}

\date{\today}

\begin{document}

\maketitle

Our first goal is to define the set \(\mathbb{R}\) of real numbers. We will \emph{assume} we know definitions of:
\begin{enumerate}[label = \textbullet, left = 0pt]
	\item the set of natural numbers \( = \mathbb{N} = \{1, 2, 3, \dots\}\)
	\item the set of integers \( = \mathbb{Z} = \{\dots, -3, -2, -1, 0, 1, 2, 3, \dots\}\)
	\item the set of rational numbers \( = \mathbb{Q} = \) the set of all ratios \(m/n\) where \(m, n\) are integers and \(n \neq 0\)
\end{enumerate}

To make the jump from \(\mathbb{Q}\) to \(\mathbb{R}\) turns out to include a subtle new idea called ``completeness.'' To explore this idea (before we define it later this week), try the following. For each of the five situations described below, provide an example as requested, or say ``NO SUCH EXAMPLE EXISTS.''

Give an example of:
\begin{enumerate}[label = (\arabic*), left = 0pt]
	\item a sequence of rational numbers that has a finite limit, with that limit being a rational number.

		The sequence 
		\( 1, \frac{1}{2}, \frac{1}{3}, \dots, \frac{1}{n}, \dots \)
		has limit \(0\).

	\item a sequence of rational numbers that has a finite limit, with that limit being an irrational number.
		
		The sequence 
		\(  \floor{\sqrt{2}}, \frac{\floor{2\sqrt{2}}}{2}, \frac{\floor{3\sqrt{2}}}{3}, \dots, \frac{\floor{n\sqrt{2}}}{n}, \dots \) 
		has limit \(\sqrt{2}\).

	\item a sequence of irrational numbers that has a finite limit, with that limit being a rational number.
		
		The sequence 
		\(  \floor{\sqrt{2}} - \sqrt{2}, \frac{\floor{2\sqrt{2}}}{2} - \sqrt{2}, \frac{\floor{3\sqrt{2}}}{3} - \sqrt{2}, \dots, \frac{\floor{n\sqrt{2}}}{n} - \sqrt{2}, \dots \) 
		has limit \(0\).
		
	\item a sequence of irrational numbers that has a finite limit, with that limit being an irrational number.

		The sequence 
		\( 1 + \sqrt{2}, \frac{1}{2} + \sqrt{2}, \frac{1}{3} + \sqrt{2}, \dots, \frac{1}{n} + \sqrt{2}, \dots \)
		has limit \(0\).

	\item a sequence of irrational numbers that has a finite limit, with that limit being an irrational number.

		NO SUCH EXAMPLE EXISTS.

\end{enumerate}

\end{document}
